%------------------------------------------------------------------------------
% Beginning of journal.tex
%------------------------------------------------------------------------------
%
% AMS-LaTeX version 2 sample file for journals, based on amsart.cls.
%
%        ***     DO NOT USE THIS FILE AS A STARTER.      ***
%        ***  USE THE JOURNAL-SPECIFIC *.TEMPLATE FILE.  ***
%
% Replace amsart by the documentclass for the target journal, e.g., tran-l.
%
\documentclass[12pt]{amsart}

\usepackage{amsrefs}
\usepackage{amssymb}
\usepackage{hyperref}
\usepackage{fancyhdr}
\usepackage{xcolor}
\usepackage{bbm}
%\usepackage{refcheck}

\newtheorem{theorem}{Theorem}
\newtheorem*{theorem*}{Theorem}
\newtheorem{lemma}[theorem]{Lemma}
\newtheorem{proposition}[theorem]{Proposition}
\newtheorem{claim}[theorem]{Claim}
\newtheorem{corollary}[theorem]{Corollary}
\newtheorem{example}[theorem]{Example}
\newtheorem{question}[theorem]{Question}
\newtheorem{remark}[theorem]{Remark}
\newtheorem{conjecture}[theorem]{Conjecture}
\newtheorem{maintheorem}{Theorem}

\theoremstyle{definition}
\newtheorem{definition}[theorem]{Definition}
\newtheorem*{definition*}{Definition}
\newtheorem*{lemma*}{Lemma}
\usepackage{graphicx}
\usepackage{pstricks, enumerate, pst-node, pst-text, pst-plot}

\numberwithin{equation}{section}
\numberwithin{theorem}{section}


\begin{document}

\title[]{Every planar graph admits a six coloring}

\author[]{Stu D. Ent}
%    Address of record for the research reported here
\address{Department of Mathematics, Massachusetts Institute of Technology}

%    Information for third author

%\thanks{}
\date{\today}


\begin{abstract}
  We show that the vertices of every planar graph can be colored by
  six colors with no two adjacent vertices sharing the same color.
\end{abstract}

\maketitle
%\tableofcontents

\section{Introduction}
Let $G = (V,E)$ be a finite, undirected, planar graph. A $k$-coloring
of $G$ is a function $f \colon V \to \{1,\ldots,k\}$ such that $f(u)
\neq f(w)$ for all $(u,w) \in E$. We say that $G$ is $k$-colorable if
it admits a $k$-coloring.

\begin{theorem}
  \label{thm:six-coloring}
  Every planar graph admits a $6$-coloring.
\end{theorem}

To prove this theorem we will use the following lemma, which was
proved in class.
\begin{lemma}
  \label{lemma:degree-five}
  In every planar graph there exists a vertex with degree at most five.
\end{lemma}


\begin{proof}[Proof of Theorem~\ref{thm:six-coloring}]
  {\bf Your proof goes here.}
\end{proof}


\end{document}

