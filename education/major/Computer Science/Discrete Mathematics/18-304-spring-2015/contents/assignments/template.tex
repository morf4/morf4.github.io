%------------------------------------------------------------------------------
% Beginning of journal.tex
%------------------------------------------------------------------------------
%
% AMS-LaTeX version 2 sample file for journals, based on amsart.cls.
%
%        ***     DO NOT USE THIS FILE AS A STARTER.      ***
%        ***  USE THE JOURNAL-SPECIFIC *.TEMPLATE FILE.  ***
%
% Replace amsart by the documentclass for the target journal, e.g., tran-l.
%
\documentclass[12pt]{amsart}

\usepackage{amsrefs}
\usepackage{amssymb}
\usepackage{hyperref}
\usepackage{fancyhdr}
\usepackage{xcolor}
\usepackage{bbm}
%\usepackage{refcheck}

\newtheorem{theorem}{Theorem}
\newtheorem*{theorem*}{Theorem}
\newtheorem{lemma}[theorem]{Lemma}
\newtheorem{proposition}[theorem]{Proposition}
\newtheorem{claim}[theorem]{Claim}
\newtheorem{corollary}[theorem]{Corollary}
\newtheorem{example}[theorem]{Example}
\newtheorem{question}[theorem]{Question}
\newtheorem{remark}[theorem]{Remark}
\newtheorem{conjecture}[theorem]{Conjecture}
\newtheorem{maintheorem}{Theorem}

\theoremstyle{definition}
\newtheorem{definition}[theorem]{Definition}
\newtheorem*{definition*}{Definition}
\newtheorem*{lemma*}{Lemma}
\usepackage{graphicx}
\usepackage{pstricks, enumerate, pst-node, pst-text, pst-plot}

\numberwithin{equation}{section}
\numberwithin{theorem}{section}


\begin{document}

\title[]{There is no fraction whose square is equal to two}

\author[]{Anne Onymous}
%    Address of record for the research reported here
\address{Somewhere in ancient Greece.}

%    Information for third author

%\thanks{}
\date{\today}


\begin{abstract}
  We show that there is no fraction whose square is equal to two.
\end{abstract}

\maketitle
%\tableofcontents

\section{Introduction}
It has long been assumed that there is some fraction $m/n$ (where $m$
and $n$ are natural numbers), such that $(m/n)^2=2$. In this paper we
show that this is in fact not the case. We accordingly propose to call
the square root of two an {\em irrational number}.


\section{Main theorem}
Before stating our theorem we will need the following standard definition.
\begin{definition}
  The {\em natural numbers} $\mathbb{N}$ are the set $\{1,2,\ldots\}$.
\end{definition}

Our main result is the following theorem.
\begin{theorem}
  \label{thm:sqrt}
  There are no two natural numbers $m$ and $n$ such that $(m/n)^2=2$.
\end{theorem}

\section{Proof}
We first state and prove a simple lemma.
\begin{lemma}
  \label{lem:sqr}
  Let $m$ be a natural number. If $m^2$ is even then it is divisible
  by $4$ and $m$ is even.
\end{lemma}
\begin{proof}
  We prove this lemma by showing that if $m$ is odd then $m^2$ is odd,
  and that if $m$ is even then $m^2$ is divisible by $4$.

  Assume first that $m$ is odd. Hence there exists a $k \in \mathbb{N}$
  such that $m = 2k-1$. Hence
  \begin{align*}
    m^2 = (2k-1)^2 = 4k^2-4k+1 = 4(k^2-k) + 1.
  \end{align*}
  Since $4(k^2=k)$ is even, it follows that $m^2$ is odd.

  Assume now that $m$ is even, so that it is equal to $2k$, where $k$
  is again some natural number. Then $m^2=4k^2$, which is
  divisible by $4$.
\end{proof}

We are now ready to prove Theorem~\ref{thm:sqrt}.
\begin{proof}[Proof of Theorem~\ref{thm:sqrt}]
  Assume by contradiction that there exist $m',n' \in \mathbb{N}$ such
  that $(m'/n')^2 = 2$. Let $k$ be the largest common divisor of $m'$ and
  $n'$, and let $m=m'/k$ and $n=n'/k$. Then
  \begin{align*}
    \left(\frac{m}{n}\right)^2 &= \left(\frac{m'/k}{n'/k}\right)^2\\
    &= \left(\frac{m'}{n'}\right)^2\\
    &= 2.
  \end{align*}
  It follows that $2n^2=m^2$, and therefore $m^2$ is even. It follows
  from Lemma~\ref{lem:sqr} that $m$ is even. It also follows from this
  lemma that $m^2$ is divisible by $4$, and so $2n^2$ is divisible by
  $4$. Hence $n^2$ is even, and, again by Lemma~\ref{lem:sqr}, $n$ is
  even. We have shown that $m$ and $n$ are both divisible by
  $2$. Hence $2k$ is a divisor of both $m'$ and $n'$, in contradiction
  to our assumption.
\end{proof}



%\bibliography{}


\end{document}

