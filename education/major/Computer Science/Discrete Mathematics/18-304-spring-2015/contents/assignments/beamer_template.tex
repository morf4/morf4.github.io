\documentclass{beamer}

\usepackage{beamerthemesplit}
\usepackage{bbm}
\setbeamertemplate{footline}[frame number]
\setlength{\unitlength}{\textwidth}
\renewcommand{\figurename}{}

\usepackage{ulem}
%\usepackage{lmodern}
\usetheme{JuanLesPins}
%\usetheme{Rochester}
%\usetheme{boxes}
\beamertemplatenavigationsymbolsempty

\newtheorem{claim}[theorem]{Claim}
\newtheorem{question}[theorem]{Question}
\newtheorem{conjecture}[theorem]{Conjecture}
\newtheorem{exercise}[theorem]{Exercise}
%\newtheorem{definition}[theorem]{Definition}



\title{There is no fraction whose square is equal to two}
\author{Anne Onymous} \date{\today}

\setbeamertemplate{headline}{}

\begin{document}
\newcounter{realtotalframenumber}{\value{totalframenumber}}
\frame{\titlepage}

\setbeamertemplate{navigation symbols}{}

\frame{
  \frametitle{The square root of two}
  \begin{itemize}
    \item It has long been assumed that $m/n =2$ for some $m,n \in
      \mathbb{N}$
    \item We show that this is in fact not
      the case!
    \item We accordingly propose to call the square root of two an
      {\em irrational number}.
  \end{itemize}

%  \begin{figure}[h]
%    \centering
%    \includegraphics{image.jpg}
%  \end{figure}
}

\frame{
  \frametitle{A definition}

  \begin{definition}
    The {\em natural numbers} $\mathbb{N}$ are the set $\{1,2,\ldots\}$.
  \end{definition}
}

  
\frame{
  \frametitle{Main theorem}
  \begin{theorem}
    \label{thm:sqrt}
    There are no two natural numbers $m$ and $n$ such that $(m/n)^2=2$.
  \end{theorem}
}

\frame{
  \frametitle{A lemma}
  \begin{lemma}
    \label{lem:sqr}
    Let $m$ be a natural number. If $m^2$ is even then it is divisible
    by $4$ and $m$ is even.
  \end{lemma}
}

\frame{
  \frametitle{Proof of lemma}
  We prove by showing that
  \begin{itemize}
  \item If $m$ is odd then $m^2$ is odd, and that
  \item if $m$ is even then $m^2$ is divisible by $4$.
  \end{itemize}
}

\frame{
  \frametitle{Proof of lemma}
  \begin{itemize}
  \item Assume first $m$ is odd.
  \item $\Rightarrow$ there exists a $k \in \mathbb{N}$ such that $m =
    2k-1$.
  \item Hence
    \begin{align*}
      m^2 = (2k-1)^2 = 4k^2-4k+1 = 4(k^2-k) + 1.
    \end{align*}
  \item $\Rightarrow$ $m^2$ is odd.
  \end{itemize}
}


\frame{
  \frametitle{Proof of lemma}
  \begin{itemize}
  \item Assume next $m$ is even.
  \item $\Rightarrow$ there exists a $k \in \mathbb{N}$ such that $m =
    2k$.

  \item $\Rightarrow \quad m^2=4k^2$.
  \item $\Rightarrow \quad m^2$ is divisible by $4$.
  \end{itemize}
}


\frame{
  \frametitle{Proof of main theorem}
  \begin{itemize}
  \item Assume by contradiction that $\exists m',n' \in \mathbb{N}$
    such that $(m'/n')^2 = 2$.
  \item Let $k$ be the largest common divisor of $m'$ and $n'$.
  \item Let $m=m'/k$ and $n=n'/k$. Then
    \begin{align*}
      \left(\frac{m}{n}\right)^2 &= \left(\frac{m'/k}{n'/k}\right)^2\\
      &= \left(\frac{m'}{n'}\right)^2\\
      &= 2.
    \end{align*}
  \end{itemize}
}
\frame{
  \frametitle{Proof of main theorem}
  \begin{itemize}
  \item $\Rightarrow\quad 2n^2=m^2$, and therefore $m^2$ is even.
  \item Applying the lemma:
    \begin{itemize}
    \item  $m$ is even, and
    \item $m^2$ is divisible by $4$.
    \end{itemize}
  \item $\Rightarrow \quad 2n^2$ is divisible by $4$, $n^2$ is even
    and so $n$ is even (lemma again).
  \end{itemize}
}

\frame{
  \frametitle{Proof of main theorem}
  \begin{itemize}
  \item We have shown that $m$ and $n$ are both divisible by
    $2$.
  \item Hence $2k$ is a divisor of both $m'$ and $n'$.
  \item This contradicts to our assumption.

  \end{itemize}
}


\end{document}
